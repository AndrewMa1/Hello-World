Java basic knowledeg 
01

Const Number:
   利用关键字 final 指示常量
   如:final double MAX=13; final表示这个变量只能被赋值一次。
   public static final .... 设置一个类常量,即可在一个类的多个方法中使用。

枚举类型(Enumeration type):
    自定义枚举类型,限制变量取值范围,如:
    enum Size{SMALL,MEDIUM,LARGE,EXTRA_LARGE}
    Size s=Size.MEDIUM.

Math function:
Math.function_name(type_name);
如:Math.sqrt(x) 计算平方根
    Math.pow(x,a),计算x的a次方
    Math.floorMod(position+adjustment,12),得到一个0-11之间的数
    Math.exp()
    Math.log10()
    Math.PI,Math.E,为两个常量,无限接近π和e。

String preliminary:
*提取子串: String greeting="Hello";
          String s=greeting.substring(0,3);
          (s=="Hel")
string_name.substring(a,b) a为起始位置,b为字符个数
*拼接:利用+号将两个字符串拼接(当讲一个字符串与非字符串的值进行拼接时,后者会转换成字符串)
*join: 用一个界定符分隔多个字符串  String.join("qualifier","string_name1","string_name2",.....);
                                               qualifier字符串为指定的界定符,需放在首位
*repeat: String repeated="Java".repeat(3); // repeated=="JavaJavaJava";
*检查字符串是否相等:s.equals(t),检查字符串s与t是否一致,若相等返回true,否则返回false。
 (若想比较的是否不区分大小写,可利用s.equalsIgnoreCase(t) )  
 (一定不要使用==来比较)
*空串与NULL串:(注意空串和NULL串是不一样的概念)
 检查空串:if(str.length())或if(str.equals(""))
 检查NULL串:if(str==NULL)
*定位代码单元:s.charAt(n); 如:char first="Hello".charAt(0),则first='H';
*比较:string1.compareTo(string2),按照字典序,如果string1在string2前面则返回一个负数,相等返回0,否则返回一个正数
***构建字符串:
   StringBuilder类与StringBuffer类
   StringBuilder string= new StringBuilder(10); //构建10字符内存的空串
   string.append("substr"); 在string尾部加入字符串substr(类比于c中的strcat);
   string.insert(1,"Java"); 在1这个位置插入字符串Java
   string.delete(a,b); //删除string中[a,b-1]的字符

输入与输出
*Scanner类:定义在java.util.*包中
    构建Scanner对象:Scanner in=new Scanner(System.in);
    in.nextline()——读取一行
    in.next()——读取一个单词(以空格作为分隔符)
    in.nextInt()——读取一个整数
*格式化输出:System.out.printf
*文件输入与输出:
    输入:构建Scanner对象:Scanner in=new Scanner(Path.of(FILE_NAME),StandardCharsets.UTF_8);
    输出:构造PrintWrite对象:PrintWrite out=new PrintWrite(FILE_NAME,StandardCharsets.UTF_8);
    (System.getProverty(FILE_NAME),找到目录的位置);(StandardCharsets.UTF_8为常用字符编码,但不同机器、网站所用的字符编码可能不同)(PrintWrite类定义在java.io.*包中)

控制流程:
*块作用域:块是指由若干条java语句组成的语句,并用一对大括号括起来。块确定了变量的作用域。
          不能在嵌套的两个块中声明同名的变量。
*带标签的break语句:将任意命名的标签放在希望跳出的最外层循环之前,且标签后紧跟一个冒号。

大数:
*java.math包中的两个类:BigInteger,BigDecimal
      将普通的数值转换为大数:BigInteger a=Biginteger.valueof(100);
      BigInteger a=new BigInteger("1231441241264364657889");
*大数的算术运算符:
      加法:add
      乘法:multiply
      除法:divide

数组:
*定义:int[] a; 
*初始化:int[] a=new int[100]; //一旦创建,则无法再改变长度(若想改变需用array list数据结构)
(在java中,允许有长度为0的数组)
*for each循环:(该循环遍历会遍历数组中的每个元素,当要处理所有数组元素时,该循环相比传统for循环更简洁)
     for(variable:collection) statement   variable为一个临时定义的变量,collection这一表达式必须是一个数组或实现了Iterable接口的类对象。
     如:for(int element:a)
         {
             System.out.println(element);
          }
     打印数组a中的每一个元素,且一个元素占一行
*数组拷贝:所有值拷贝:Arrays.copyOf(array_name,length); //这个方法也可以用来增加数组的大小
*数组排序:Arrays.sort();
*其他方法:
     static void fill(xxx[] a,xxx v); 将数组a中所有数据元素置为v
     static int binarySearch(xxx[] a,int start,int end,xxx v);
     static boolean equals(xxx[] a,xxx[] b);














     

